\documentclass{article}
\usepackage[paperwidth=180mm,paperheight=200mm,left=10mm,right=10mm,top=10mm,bottom=10mm]{geometry}
\usepackage{graphicx}
\usepackage{color}
\usepackage{listings}
\setlength{\parskip}{2mm}
\setlength{\parindent}{0mm}
 \usepackage{lmodern}
 \usepackage{ulem}
\usepackage[T1]{fontenc}
\title{Control problem modifications that attempt to reduce mass imbalance}
\author{S L Cornford}
\date{3 Jun 2013}\newcommand{\usrf}{s}
\newcommand{\topg}{r}
\newcommand{\thk}{h}

\lstset{language=C++,basicstyle=\scriptsize,commentstyle=\color{red}
}


\begin{document}
\pagestyle{plain}
\maketitle
\tableofcontents

\section{Mass balance problem}

It is easy enough to find $C$ and $\phi$ in the convecntional control problem such that the
observed and model velocity are well-matched. But when we do so, we find that $\vec{\nabla}. (\vec{u} H)$
is far too large, and noisy (fig \ref{fig::divuh10}. Around the Pine Island grounding line, we see $\vec{\nabla}. (\vec{u} H) \sim 100~\mbox{m/a}$,
that is, a huge thickneing rate. Previously, we have attempted to relax the ice sheet after the control problem has done, and in fact had to
introuduce extra softeness (through a reduced $phi$) around the PIG GL.  

Ideally, we woould like to modify the control problem to minimize both $||~|\vec{u}|_{\rm obs} - |\vec{u}|~||$ 
\emph{and} $||\vec{\nabla}. (\vec{u} H)_{\rm obs}) - \vec{\nabla}. (\vec{u} H)  ||$, or some linear combination
of them. In earlier work, we attempted to do just that, but it became apparent that it is the topography 
(and ice thickness) that need to be found. TODO : check Morlighem and Perego work on this area, Morloghem 
looks especially promising and I think is similar to the approach of section \ref{sec::toporelax}.

\begin{figure}
\begin{center}
\includegraphics[width=0.45\textwidth]{ase-mass-balance.gfx/ase-test0-ctrl-0002.png}
\includegraphics[width=0.45\textwidth]{ase-mass-balance.gfx/ase-test0-ctrl-0001.png}
\end{center}
\caption{\protect{\label{fig::divuh10}}} flux divergence $\vec{\nabla}. (\vec{u} H)$
after 10 CG iterations. Around the Pine Island grounding line, we see $\vec{\nabla}. (\vec{u} H) \sim 100~\mbox{m/a}$
\end{figure}

\section{First thoughts}

Choosing $H$ to minimize  $||\vec{\nabla}. (\vec{u} H)_{\rm obs}) - \vec{\nabla}. (\vec{u} H)  ||$ isn't
epsecially easy because there is not an obvious norm (but check Schoof 2006) which leads to the
advection equation. Obvious norms, e.g the L2-norm $||\vec{\nabla}. (\vec{u} H)_{\rm obs}) - \vec{\nabla}. (\vec{u} H)  ||^2_2$ lead 
in essence to diffusions equations. On top of that, the advection equation operator $A$ is not symettric positive definite,
so simply using it in the CG gradient (ie attempting to exploit a descent direction $H \leftarrow H - \alpha  \vec{\nabla}. (\vec{u} H)$)
end up solving something like $ (A + A^T)h = 2 r$ rather than $ A H = r$ -- ie a diffusion equation rather than an advection equation.
I tried both these out anyway, without decent results.

\section{\label{sec::toporelax} Relaxation of topography}

Last year I tried relaxing the topgraphy by evolving the ice sheet and holding the upper surface and the ice shelf constant. 
This looked prosmising, but it was a bit slow, and produced annoyoin results. e.g moving the PIG
GL, and I abandoned it for the time being. However, I saw a talk by Mathieu Morlighem where he had acheive
good results in this sort of way for Greenland and PIG, so it makes sense to retry the idea. This time around, 
I have attempted to embed the topgraphy relaxtion in the control problem code (control.cpp) rather than rely on the
usual time dependent code (driver.cpp).

The basic idea is
\begin{enumerate}
\item{Carry out $n$ CG iterations of the control problem}
\item{Evolve the ice thickness for $T$ years through, $H = H + \Delta t ( \vec{\nabla}. (\vec{u} H)_{\rm obs}) = \vec{\nabla}. (\vec{u} H)$),
  holding the upper surface constant (ie lowering the bed when the ice thickens) }
\item{Repeat 1 and 2 till happiness ensures}
\end{enumerate}
Note that in step 2 I evolve the ice thickness holding $u(x,y)$ constant throughout
in time. That reduces the computational cost considerably, as the time dependent problem 
is domiated by velocity solves.

\subsection{Test 0}


I set $n$ = 10 and $\Delta t = 0.01$ and $T=1$. In the ice sheet, I set 
$\vec{\nabla}. (\vec{u} H)_{\rm obs}) = 0$ - in practice this should $a_{\rm obs} - H'_{\rm obs}$.
In the ice shelf,  I set $\vec{\nabla}. (\vec{u} H)_{\rm obs}) = \vec{\nabla}. (\vec{u} H)$,
which has the effect of holding its geoemtry consstant (not ideal, but we don't want $\vec{\nabla}. (\vec{u} H) = 0$, and we don't
know the melt-rate). Also, in order to avoid recomputing the shelf/sheet mask (and hence moving the region 
in which $\vec{\nabla}. (\vec{u} H)_{\rm obs}) = \vec{\nabla}. (\vec{u} H)$), I don't call LevelSigmaCS::recomputeGeoemtry(), 
which means that the face-centered thickness doesn't get updated (and should). 

\begin{verbatim}
URL: https://anag-repo.lbl.gov/svn/BISICLES/public/branches/inverseproblem/code/controlproblem
Repository Root: https://anag-repo.lbl.gov/svn/BISICLES
Repository UUID: 4b1697f3-9003-4f1e-b6a8-e72787e6a227
Revision: 2157
Node Kind: directory
Schedule: normal
Last Changed Author: slcornford
Last Changed Rev: 2158
Last Changed Date: 2013-08-19 13:34:08 +0100 (Mon, 19 Aug 2013)
\end{verbatim}


\begin{figure}
\begin{center}
\includegraphics[width=0.45\textwidth]{ase-mass-balance.gfx/ase-test0-ctrl-0005.png}
\includegraphics[width=0.45\textwidth]{ase-mass-balance.gfx/ase-test0-ctrl-0006.png}
\end{center}
\caption{\protect{\label{fig::divuh10}}} flux divergence $\vec{\nabla}. (\vec{u} H)$
after 80 CG iterations with topography evolution every 10 iterations. 
Around the Pine Island grounding line, we see $\vec{\nabla}. (\vec{u} H) \sim 100~\mbox{m/a}$ has
beed reduced a great deal, though there is now a thickness step there and a huge (km/a) melt rate
implied in the ice shelf.
\end{figure}

{\large {\bf Conclusions:}} The basic idea seems to work, but I need to allow the thickness to evolve in the ice shelf, and
update the face-centered thickness in LevelSigmaCS. It would also make sense to carry out relaxtions based on the misfit rather
than simply every 10 iterations, presumably so that at the end of the problem only a few CG iterations are needed 
between each relaxation.


\bibliographystyle{unsrt}
\bibliography{Antarctica2013}
\end{document}
