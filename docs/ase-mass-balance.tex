\documentclass{article}
\usepackage[paperwidth=180mm,paperheight=200mm,left=10mm,right=10mm,top=10mm,bottom=10mm]{geometry}
\usepackage{graphicx}
\usepackage{color}
\usepackage{listings}
\setlength{\parskip}{2mm}
\setlength{\parindent}{0mm}
 \usepackage{lmodern}
 \usepackage{ulem}
\usepackage[T1]{fontenc}
\title{Control problem modifications that attempt to reduce mass imbalance}
\author{S L Cornford}
\date{3 Jun 2013}\newcommand{\usrf}{s}
\newcommand{\topg}{r}
\newcommand{\thk}{h}

\lstset{language=C++,basicstyle=\scriptsize,commentstyle=\color{red}
}


\begin{document}
\pagestyle{plain}
\maketitle
\tableofcontents

\section{Mass balance problem}

It is easy enough to find $C$ and $\phi$ in the convectional control problem such that the
observed and model velocity are well-matched. But when we do so, we find that $\vec{\nabla}. (\vec{u} H)$
is far too large, and noisy (fig \ref{fig::divuh10}. Around the Pine Island grounding line, we see $\vec{\nabla}. (\vec{u} H) \sim 100~\mbox{m/a}$,
that is, a huge thickening rate. Previously, we have attempted to relax the ice sheet after the control problem has done, and in fact had to
introduce extra softness (through a reduced $phi$) around the PIG GL.  

Ideally, we would like to modify the control problem to minimize both $||~|\vec{u}|_{\rm obs} - |\vec{u}|~||$ 
\emph{and} $||\vec{\nabla}. (\vec{u} H)_{\rm obs}) - \vec{\nabla}. (\vec{u} H)  ||$, or some linear combination
of them. In earlier work, we attempted to do just that, but it became apparent that it is the topography 
(and ice thickness) that need to be found. TODO : check Morlighem and Perego work on this area, Morloghem 
looks especially promising and I think is similar to the approach of section \ref{sec::toporelax}.

\begin{figure}
\begin{center}
\includegraphics[width=0.45\textwidth]{ase-mass-balance.gfx/ase-test0-ctrl-0002.png}
\includegraphics[width=0.45\textwidth]{ase-mass-balance.gfx/ase-test0-ctrl-0001.png}
\end{center}
\caption{\protect{\label{fig::divuh10}}} flux divergence $\vec{\nabla}. (\vec{u} H)$
after 10 CG iterations. Around the Pine Island grounding line, we see $\vec{\nabla}. (\vec{u} H) \sim 100~\mbox{m/a}$
\end{figure}

\section{First thoughts}

Choosing $H$ to minimize  $||\vec{\nabla}. (\vec{u} H)_{\rm obs}) - \vec{\nabla}. (\vec{u} H)  ||$ isn't
especially easy because there is not an obvious norm (but check Schoof 2006) which leads to the
advection equation. Obvious norms, e.g the L2-norm $||\vec{\nabla}. (\vec{u} H)_{\rm obs}) - \vec{\nabla}. (\vec{u} H)  ||^2_2$ lead 
in essence to diffusion equations. On top of that, the advection equation operator $A$ is not symmetric positive definite,
so simply using it in the CG gradient (ie attempting to exploit a descent direction $H \leftarrow H - \alpha  \vec{\nabla}. (\vec{u} H)$)
end up solving something like $ (A + A^T)h = 2 r$ rather than $ A H = r$ -- ie a diffusion equation rather than an advection equation.
I tried both these out anyway, without decent results.

\section{\label{sec::toporelax} Relaxation of topography}

Last year I tried relaxing the topography by evolving the ice sheet and holding the upper surface and the ice shelf constant. 
This looked promising, but it was a bit slow, and produced annoying results. e.g moving the PIG
GL, and I abandoned it for the time being. However, I saw a talk by Matthieu Morlighem where he had achieve
good results in this sort of way for Greenland and PIG, so it makes sense to retry the idea. This time around, 
I have attempted to embed the topography relaxation in the control problem code (control.cpp) rather than rely on the
usual time dependent code (driver.cpp).

The basic idea is
\begin{enumerate}
\item{Carry out $n$ CG iterations of the control problem}
\item{Evolve the ice thickness for $T$ years through, $H = H + \Delta t ( \vec{\nabla}. (\vec{u} H)_{\rm obs}) = \vec{\nabla}. (\vec{u} H)$),
  holding the upper surface constant (ie lowering the bed when the ice thickens) }
\item{Repeat 1 and 2 till happiness ensures}
\end{enumerate}
Note that in step 2 I evolve the ice thickness holding $u(x,y)$ constant throughout
in time. That reduces the computational cost considerably, as the time dependent problem 
is dominated by velocity solves.

\subsection{Test 0}


I set $n$ = 10 and $\Delta t = 0.01$ and $T=1$. In the ice sheet, I set 
$\vec{\nabla}. (\vec{u} H)_{\rm obs}) = 0$ - in practice this should $a_{\rm obs} - H'_{\rm obs}$.
In the ice shelf,  I set $\vec{\nabla}. (\vec{u} H)_{\rm obs}) = \vec{\nabla}. (\vec{u} H)$,
which has the effect of holding its geometry constant (not ideal, but we don't want $\vec{\nabla}. (\vec{u} H) = 0$, and we don't
know the melt-rate). Also, in order to avoid recomputing the shelf/sheet mask (and hence moving the region 
in which $\vec{\nabla}. (\vec{u} H)_{\rm obs}) = \vec{\nabla}. (\vec{u} H)$), I don't call LevelSigmaCS::recomputeGeoemtry(), 
which means that the face-centered thickness doesn't get updated (and should). 

\begin{verbatim}
URL: https://anag-repo.lbl.gov/svn/BISICLES/public/branches/inverseproblem/code/controlproblem
Repository Root: https://anag-repo.lbl.gov/svn/BISICLES
Repository UUID: 4b1697f3-9003-4f1e-b6a8-e72787e6a227
Revision: 2157
Node Kind: directory
Schedule: normal
Last Changed Author: slcornford
Last Changed Rev: 2158
Last Changed Date: 2013-08-19 13:34:08 +0100 (Mon, 19 Aug 2013)
\end{verbatim}


\begin{figure}
\begin{center}
\includegraphics[width=0.45\textwidth]{ase-mass-balance.gfx/ase-test0-ctrl-0005.png}
\includegraphics[width=0.45\textwidth]{ase-mass-balance.gfx/ase-test0-ctrl-0006.png}
\end{center}
\caption{\protect{\label{fig::divuh10}}} test 0 flux divergence $\vec{\nabla}. (\vec{u} H)$
after 80 CG iterations with topography evolution every 10 iterations. 
Around the Pine Island grounding line, we see $\vec{\nabla}. (\vec{u} H) \sim 100~\mbox{m/a}$ has
been reduced a great deal, though there is now a thickness step there and a huge (km/a) melt rate
implied in the ice shelf.
\end{figure}

{\large {\bf Conclusions:}} The basic idea seems to work, but I need to allow the thickness to evolve in the ice shelf, and
update the face-centered thickness in LevelSigmaCS. It would also make sense to carry out relaxations based on the misfit rather
than simply every 10 iterations, presumably so that at the end of the problem only a few CG iterations are needed 
between each relaxation.

\newpage

\subsection{Test 1}

Face-centered thickness now updated in LevelSigmaCS.

Allowing thickness to evolve in the ice shelf. On each time-step I compute
\begin{equation}
\Delta H = - \Delta t ( \vec{\nabla}. (\vec{u} H) - \vec{\nabla}. (\vec{u} H) _{\rm obs}).
\end{equation}
But rather than modify the topography with $\Delta r = -\Delta H$ in the ice sheet only, I compute 
\begin{equation}
  \Delta r = \left \{ 
  \begin{array}{lr} 
    {\rm max} (-\Delta H,  -(H + \Delta H)  \rho_{\rm i}/\rho_{\rm w} - r) & \mbox{for grounded ice} \\
    {\rm min} (-\Delta H,  -(H + \Delta H) \rho_{\rm i}/\rho_{\rm w} - r)& \mbox{for floating ice} \\
    0 & \mbox{elsewhere}
  \end{array} \right. 
\end{equation}
which preserves the mask, and provided the mask is preserved, the surface elevation in the ice sheet.

On every CG restart (every 10 iterations in this case), there is an evolution stage (as before), but now I compute
a melt-rate $-M_b$ at the start of each new evolution. I attempt to find a smooth $M_b$ based on the flux divergence
in the ice shelf by solving 
\begin{equation}
  M_b - \lambda^2 \nabla ^2 M_b = - \left \{ 
  \begin{array}{lr} 
    \vec{\nabla}. (\vec{u} H) & \mbox{for floating ice} \\
    0 & \mbox{elsewhere}
  \end{array} \right. 
\end{equation}

\begin{verbatim}
URL: https://anag-repo.lbl.gov/svn/BISICLES/public/branches/inverseproblem/code
Repository Root: https://anag-repo.lbl.gov/svn/BISICLES
Repository UUID: 4b1697f3-9003-4f1e-b6a8-e72787e6a227
Revision: 2170
Node Kind: directory
Schedule: normal
Last Changed Author: slcornford
Last Changed Rev: 2164
Last Changed Date: 2013-08-20 17:05:02 +0100 (Tue, 20 Aug 2013)
\end{verbatim}

some output and inputs files stored:
\begin{verbatim}
[ggslc@porthos run]$ cp ase-test0-ctrl-2lev-outer.0000?0.2d.hdf5 
/data/ggslc/BISICLES-Antarctica2013/control/ase/mass-balance/test1/
[ggslc@porthos run]$ cp inputs.testm 
/data/ggslc/BISICLES-Antarctica2013/control/ase/mass-balance/test1/
\end{verbatim}

\begin{figure}
\begin{center}
\includegraphics[width=0.45\textwidth]{ase-mass-balance.gfx/ase-test1-ctrl-0008.png}
\includegraphics[width=0.45\textwidth]{ase-mass-balance.gfx/ase-test1-ctrl-0009.png}
\end{center}
\caption{\protect{\label{fig::test1a}}} test 1 flux divergence $\vec{\nabla}. (\vec{u} H)$
after 10 CG iterations. 
\end{figure}

\begin{figure}
\begin{center}
\includegraphics[width=0.45\textwidth]{ase-mass-balance.gfx/ase-test1-ctrl-0010.png}
\includegraphics[width=0.45\textwidth]{ase-mass-balance.gfx/ase-test1-ctrl-0011.png}
\end{center}
\caption{\protect{\label{fig::test1b}}} test 1  flux divergence $\vec{\nabla}. (\vec{u} H)$
after 43 CG iterations. Smoother, much reduced.
\end{figure}

\begin{figure}
\begin{center}
\includegraphics[width=0.45\textwidth]{ase-mass-balance.gfx/ase-test1-ctrl-0012.png}
\includegraphics[width=0.45\textwidth]{ase-mass-balance.gfx/ase-test1-ctrl-0013.png}
\end{center}
\caption{\protect{\label{fig::test1c}}} test 1 flux divergence $\vec{\nabla}. (\vec{u} H)$
after 83 CG iterations. Smoother, much reduced.
\end{figure}


\begin{figure}
\begin{center}
\includegraphics[width=0.75\textwidth]{ase-mass-balance.gfx/ase-test1-ctrl-0014.png}
\end{center}
\caption{\protect{\label{fig::test1c}}} test 1  flux divergence $\vec{\nabla}. (\vec{u} H)$
after 83 CG iterations. Smoother, much reduced.
\end{figure}

{\large {\bf Conclusions:}} This all seems to be an improvement. Notably, however, PIG is no longer grounded on a ridge
(or rather, the ridge has been eroded). Need to try the (slower, earlier) Joughin velocities and consider allowing the
GL to move or imposing a new GL from (say) Shepherd. Also $M_b$ to be computed on the first evolution and held constant in time.

\subsection{Test 2}

As test 1, but with $M_b$ computed on the first evolution and held constant in time.

\begin{figure}
\begin{center}
\includegraphics[width=0.45\textwidth]{ase-mass-balance.gfx/ase-test2-ctrl-0000.png}
\includegraphics[width=0.45\textwidth]{ase-mass-balance.gfx/ase-test2-ctrl-0001.png}
\end{center}
\caption{\protect{\label{fig::test1b}}} test 2 flux divergence $\vec{\nabla}. (\vec{u} H)$
after 10 and 43 CG iterations. $M_b$ computed on the first evolution and held constant in time.
Slightly different transect from previous figures (VisIt restarted).
\end{figure}


{\large {\bf Conclusions:}} This all seems to be an improvement. Notably, however, PIG is no longer grounded on a ridge
(or rather, the ridge has been eroded). Need to try the (slower, earlier) Joughin velocities and consider allowing the
GL to move or imposing a new GL from (say) Shepherd. Also $M_b$ to be computed on the first evolution and only decreased 
(more melting) in subsequent stages. 

\subsection{Test 3}

As test 1, but with $M_b$ computed on the first evolution and thereafter only increased. That is
$M_b(t) = \mbox{min}(M_b(t = 0), L^{-1} \vec{\nabla}. (\vec{u} H))$, with $L$ the Poisson operator
$L v = v - \lambda^2 \nabla^2 v$ .

\begin{figure}
\begin{center}
\includegraphics[width=0.45\textwidth]{ase-mass-balance.gfx/ase-test3-ctrl-0000.png}
\includegraphics[width=0.45\textwidth]{ase-mass-balance.gfx/ase-test3-ctrl-0001.png}
\end{center}
\caption{\protect{\label{fig::test1b}}} test 3 initial topography and transect.
Added the lowest bed elevation for grounded ice given the surface elevation 
$s - s/(1-\rho _{\rm i}/\rho _{\rm w}$ (rmin)
\end{figure}

\begin{figure}
\begin{center}
\includegraphics[width=0.45\textwidth]{ase-mass-balance.gfx/ase-test3-ctrl-0002.png}
\includegraphics[width=0.45\textwidth]{ase-mass-balance.gfx/ase-test3-ctrl-0003.png}
\end{center}
\caption{\protect{\label{fig::test1b}}} test 3 CG 90 topography and transect.
Added the lowest bed elevation for grounded ice given the surface elevation.
\end{figure}

{\large {\bf Conclusions:} Ice is still rather thick at the GL.}

\subsection{Test 4}

As test 3, but now the grounding line can retreat - which allows the extent of melting to be modified as well
as the amplitude.



\bibliographystyle{unsrt}
\bibliography{Antarctica2013}
\end{document}
