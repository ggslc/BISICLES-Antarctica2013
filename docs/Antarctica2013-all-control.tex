\documentclass{article}
\usepackage[paperwidth=180mm,paperheight=200mm,left=10mm,right=10mm,top=10mm,bottom=10mm]{geometry}
\usepackage{graphicx}
\usepackage{color}
\usepackage{listings}
\setlength{\parskip}{2mm}
\setlength{\parindent}{0mm}
 \usepackage{lmodern}
\usepackage[T1]{fontenc}
\title{Antarctica 2013 (all Antarctica) control problem}
\author{S L Cornford}
\date{3 Jun 2013}

\newcommand{\usrf}{s}
\newcommand{\topg}{r}
\newcommand{\thk}{h}

\lstset{language=Fortran,basicstyle=\scriptsize,commentstyle=\color{red}
}


\begin{document}
\pagestyle{plain}
\maketitle
\tableofcontents
\section{Data Sources}
\subsection{Geometry}

Thickness and topography data are taken from Bedmap2\cite{FPVB13:doi:10.5194/tc-7-375-2013}. The Bedmap2 data are provided
on a $\Delta x = 1$~km, 6777 $\times$ 6777 grid centered on the South Pole, and are stored in
\begin{verbatim}
Antarctica2013/bedmap2_data/bedmap2_bin
\end{verbatim}

We need to set bedmap2 on a multigrid-friendly domain. I chose a 6144$\times$6144 domain, which
includes the entire continental shelf. The program makenc.f90 is used to this end, it also does a little
bit of preprocessing (but more is done in the programs described in section \ref{sec::preprocessing})

{\large {\bf update  4 Jun 2013}}

Removed the pre-processing from makenc.f90, now it simply subsets and writes the fields
topg (topography), thk(thickness), usrf (surface elevation), mask (land ice / ice shelf mask).
All pre-processing of thickness, topography is to be carried out in section \ref{sec::preprocessing}.

%\begin{itemize}
%\item{Lake Vostok's $\sim 800$~m deep water is removed and replaced with ice}
\subsubsection{Crop domain}

\begin{lstlisting}
integer, parameter :: ewn=6667,nsn = 6667,nc=4, unit = 16, & 
       n = 6144, ewmin= 263 ,nsmin = 263 ,ewmax=ewmin+n-1,nsmax=nsmin+n-1
!etc
  open(unit,file='bedmap2_bin/bedmap2_bed.flt', & 
  ACCESS='STREAM', FORM='UNFORMATTED')
  read(unit) sp
  close(unit)
  do i = 0,n-1
     dp(:,n-i,1) = sp(ewmin:ewmax,i+nsmin)
  end do
  field(1)="topg"
!etc
\end{lstlisting}

\subsubsection{Remove Lake Vostok}
Lake Vostok's $\sim 800$~m deep water is removed and replaced with rock
\begin{lstlisting}
  ia = 4610 - 400 - ewmin
  ib = ia + 800
  ja = 2974 - 400 - nsmin
  jb = ja + 800
  where ( (dp(ia:ib,ja:jb,1) + dp(ia:ib,ja:jb,2)) .lt.  bmusrf(ia:ib,ja:jb))
     dp(ia:ib,ja:jb,1) = bmusrf(ia:ib,ja:jb) - dp(ia:ib,ja:jb,2) 
     !dp(:,:,1) stores topography
  end where
\end{lstlisting}

\subsubsection{Correction where $\topg + \thk > \usrf$.} 
\begin{lstlisting}
  !a few spots where topography + thickness > surface : reduce thickness
  where ( (dp(:,:,1) + dp(:,:,2)) .gt.  bmusrf + eps)
      dp(:,:,2) = bmusrf - dp(:,:,1)
      !dp(:,:,2) stores thickness
  end where
\end{lstlisting}

\subsubsection{Adjust thickness so that ice shelves float}
\begin{lstlisting}
  real (kind=8), parameter :: rhoi = 918.0d0, rhoo = 1028.0d0
  !etc
 !ensure that the ice shelves (as specified by the bedmap2 mask) are floating
  where ( ( dp(:,:,2) .gt. eps) .and. ( dp(:,:,4) .gt. 0.9))
     dp(:,:,2) = min(dp(:,:,2), -rhoo/rhoi*dp(:,:,1) -1.0)
  end where
\end{lstlisting}

The thickness and topography data is written to {\tt Antarctica2013/bedmap2\_data/bedmap2-1km.nc}



\subsection{Velocity}

Velocity data are taken from Rignot's NSIDC 2011 data \cite{RiMS11:doi:10.1126/science.1208336}.
The raw data, stored in
\begin{verbatim}
Antarctica2013/rignot_velocity_data/Antarctica_ice_velocity.nc
\end{verbatim}
is sited on a 6233 $\times$ 6233 grid with  $\Delta x = 0.9$~km, and so needs to be interpolated
onto a 6144$\times$6144, $\Delta x = 1$~km domain. This is done in the program  rignot.f90, which
produces the file
\begin{verbatim}
Antarctica2013/rignot_velocity_data/Antarctica_ice_velocity-1km.nc
\end{verbatim}
For now, I only keep the modulus of velocity.

\subsection{Temperature}

Temperature data is the same as we used Antarctica2012. It is taken from Pattyn (2010), and
is stored in the ALBMAP grid in the file Rupert provided:
\begin{verbatim}
Antarctica2013/albmap_data/ALBMAP_i2s_4BISICLES.nc
\end{verbatim}
We need to place this on a grid that is compatible with the 1~km bedmap2 grid, ie a 
grid where \\ $\Delta x \in \left \{ \ldots, \frac{1}{2}, 1, 2, 4,\ldots \right \}$. I chose 
$\Delta x 4$~km to save memory. The program
albtobed.f90 carries this out and produces the file
\begin{verbatim}
Antarctica2013/albmap_data/Antarctica-temperature-4km.nc
\end{verbatim}
There is some preprocessing in this program too, the temperature is first
grown into the sea (subroutine growtemp) then modified such that it cannot increase
by more than 5 K between layers, then smoothed (subroutine sanetemp)

\subsection{Sector map}

The same program, albtobed.f90, writes a 1 km resolution sector map 
\begin{verbatim}
Antarctica2013/albmap_data/Antarctica_sectors-1km.nc
\end{verbatim}

\section{\label{sec::preprocessing}Preprocessing}

The majority of the preprocessing takes place in
\begin{verbatim}
Antarctica2013/prep_control/prepcontrol.f90
\end{verbatim}
The script
\begin{verbatim}
Antarctica2013/prep_control/prep_antarctica.sh
\end{verbatim}
builds and runs this program, and converts its netcdf output into the hdf5 format that 
BISICLES can load.

Program prepcontrol.f90 takes a namelist argument, which specifies the inputs files, output files, which regions
are to be quiescent. For the whole of Antarctica, the namelist
antarctica.nml was used.
\begin{verbatim}
!input files
&files
 ingeofile='bedmap2-1km.nc'
 invelfile='Antarctica_ice_velocity-1km.nc'
 insecfile='Antarctica_sectors-1km.nc'
 intempfile='Antarctica-temperature-4km.nc'
 outgeofile='antarcticabm2-geometry-1km.nc'
 outctrlfile='antarcticabm2-ctrldata-1km.nc'
 outsecfile='anatrcticabm2-sectors-1km.nc'
 outtempfile='antarcticabm2-temperature-4km.nc'
/

&dims
 ixlo=1
 iylo=1
 nx=6144
 ny=6144
 nsigma=10
/

!no quiescent regions
&quiescent

/
\end{verbatim}

Preprocessing consists of several modifications to the topography and thickness
fields, and an initial guess for the basal traction coeffcient $C$, called $C_0$.

\subsection{\label{sec::limthk} Correction where $\topg + \thk > \usrf$.}

There are a few spots, e.g around $(x,y) = (-36,1993)$~km, where $\topg + \thk > \usrf$.
Reduce the thickness in those regions
\begin{lstlisting}
!ensure thk + topg <= usrf... 
  where (thk + topg .gt. usrf + cm)
     thk = usrf - topg
  end where
\end{lstlisting}

\subsection{\label{sec::limthk} Revised modification to shelves.}

The original modification (reduce thickness to presrve mask)
\begin{lstlisting}
  where ( ( thk .gt. cm) .and. ( mask .gt. 0.9))
     thk = min(thk, -rhoo/rhoi*topg -1.0)
  end where
\end{lstlisting}
worked poorly in Amery Ice Shelf

An alternate modification lowers the topography to preserve 
the mask provided that the resulting surface does not
change too much
\begin{lstlisting}
!try to preserve mask, but limit change in surface
  where ( (mask.gt.0.99) .and. ( abs ( usrf - thk*(1.0d0-rhoi/rhoo)) .le. 30.0))
     topg = min(topg, -20.0 - rhoo/rhoi*thk
  end where
\end{lstlisting}

\subsection{\label{sec::rlv} Remove Lake Vostok}

Lake Vostok is filled with 800m deep water, but we don't really support subglacial
lakes (water above sea-level). So, fill it with rock.


\begin{lstlisting}
  subroutine removelakevostok(thk,topg,usrf,x,y,nx,ny)
    integer, intent(in) :: nx,ny
    real(kind=8), dimension(1:nx,1:ny), intent(inout) :: thk,topg,usrf
    real(kind=8), dimension(1:nx), intent(in) :: x
    real(kind=8), dimension(1:ny), intent(in) :: y

    integer :: i,j,ilo,jlo,ihi,jhi,ncells
    real(kind=8) :: dx,r

    dx = x(2) - x(1)
    r = rhoo/rhoi
    call computebox(ncells,ilo,ihi,jlo,jhi, & 
         1.2d+6,-0.4d+6,1.46d+6,-0.3d+6,x,y,nx,ny)
    if (ncells.gt.0) then
       topg(ilo:ihi,jlo:jhi) = usrf(ilo:ihi,jlo:jhi)-thk(ilo:ihi,jlo:jhi)
    end if

  end subroutine removelakevostok
\end{lstlisting}


\subsection{\label{sec::rtt} Remove thin tongues}
There are three thin ice tongues that cause the solvers to stagnate (Drygalski, Mertz, and one 
more in Lutzow-Holm bay). Since they have no influence in the upstream ice, I remove them

\begin{lstlisting}
  subroutine computebox(ncells,ilo,ihi,jlo,jhi,xmin,ymin,xmax,ymax,x,y,nx,ny)
  !find the largest box such that xmin < x < xmax and ymin < y < ymax
  !which fits inside the box x(1) < x < x(nx) and y(1) < y < y(ny)
  !...
  return
  end subroutine computebox
  
   subroutine removethintongues(thk,topg,umod,x,y,nx,ny)
    !there are three thin floating tonugues which cause the solvers
    !to converge slowly. Since they don't affect the upstream 
    !dynamics, snip them off

    
    integer, intent(in) :: nx,ny
    real(kind=8), dimension(1:nx,1:ny), intent(inout) :: thk,topg,umod
    real(kind=8), dimension(1:nx), intent(in) :: x
    real(kind=8), dimension(1:ny), intent(in) :: y

    integer :: i,j,ilo,jlo,ihi,jhi,ncells
    real(kind=8) :: dx,r
    dx = x(2) - x(1)
    r = (1-rhoi/rhoo)

  !Tongue 1, East Antarctica near to Lutzow-Holm bay
  call computebox(ncells,ilo,ihi,jlo,jhi,1.35d+6,1.775d+6,1.4d+6,2.075d+6,x,y,nx,ny)
  if (ncells.gt.0) then
    where (r * thk(ilo:ihi,jlo:jhi).gt.(thk(ilo:ihi,jlo:jhi) + topg(ilo:ihi,jlo:jhi))) 
      thk(ilo:ihi,jlo:jhi) = 0.0d0
    end where
  end if
  
  !Tongue 2, East Antarctica Near to East Adelie land (Mertz tongue)   
  call computebox(ncells,ilo,ihi,jlo,jhi,1.39d+6,-2.14d+6,1.46d+6,-2.05d+6,x,y,nx,ny)
  if (ncells.gt.0) then
     where (r * thk(ilo:ihi,jlo:jhi).gt.(thk(ilo:ihi,jlo:jhi) + topg(ilo:ihi,jlo:jhi))) 
        thk(ilo:ihi,jlo:jhi) = 0.0d0
     end where
  end if

  !Tongue 3, East Antarctic near to McMurdo (Drygalski tongue)
  call computebox(ncells,ilo,ihi,jlo,jhi,0.36d+6,-1.55d+6,0.46d+6,-1.25d+6,x,y,nx,ny)
  if (ncells.gt.0) then
     where (r * thk(ilo:ihi,jlo:jhi).gt.(thk(ilo:ihi,jlo:jhi) + topg(ilo:ihi,jlo:jhi))) 
        thk(ilo:ihi,jlo:jhi) = 0.0d0
     end where
  end if

  end subroutine removethintongues

\end{lstlisting}


\subsection{\label{sec::fts} Fix Thwaites' shelf}

Like ALBMAP, Bedmap2 does not have a grounded tip on Thwaites' Eastern (slow moving) ice shelf.
Such a grounded region is implied by the velcity fields, which slow down in that region - impossible
without some sort of buttress. As before, we ground the tip. We also remove some `hangnails' on the
western side of the fast tongue, these were problematic in Antarctica2012
\begin{lstlisting}

  subroutine fixthwaitesshelf(thk,topg,umod,x,y,nx,ny)
    !bedmap2 has thaites shelf free-floating, but
    !the rignot velocity data has it grounded at the tip
    !Also, remove a hangnail

    integer, intent(in) :: nx,ny
    real(kind=8), dimension(1:nx,1:ny), intent(inout) :: thk,topg,umod
    real(kind=8), dimension(1:nx), intent(in) :: x
    real(kind=8), dimension(1:ny), intent(in) :: y

    integer :: i,j,ilo,jlo,ihi,jhi,ncells
    real(kind=8) :: dx,r

    dx = x(2) - x(1)
    r = rhoo/rhoi

    !ground tip   
    call computebox(ncells,ilo,ihi,jlo,jhi,-1.598d+6,-0.458d+6,-1.586d+6,-0.446d+6,x,y,nx,ny)
    if (ncells.gt.0) then

       do j = jlo,jhi
          do i = ilo,ihi
             if ( (umod(i,j).gt.1.0d0) .and. (umod(i,j).lt.300.0d0)) then
                if (thk(i,j).gt.1.0d0) then
                   if (topg(i,j).gt.-800.0d0) then
                      topg(i,j) = max(topg(i,j),(-300.0d0*0.750d0 + topg(i,j)*0.25))
                      thk(i,j) = max(thk(i,j),-r*topg(i,j) + 20.0d0)
                   end if
                end if
             end if
          end do
       end do
    end if

    !hangnails
    call computebox(ncells,ilo,ihi,jlo,jhi,-1.565d+6,-0.512d+6,-1.545d+6,-0.502d+6,x,y,nx,ny)
    if (ncells.gt.0) then
       write(*,*) ncells,ilo,ihi,jlo,jhi
       thk(ilo:ihi,jlo:jhi) = 0.0d0
    end if

    call computebox(ncells,ilo,ihi,jlo,jhi,-1.602d+6,-0.479d+6,-1.597d+6,-0.473d+6,x,y,nx,ny)
    if (ncells.gt.0) then
       thk(ilo:ihi,jlo:jhi) = 0.0d0
    end if

  end subroutine fixthwaitesshelf
\end{lstlisting}

\subsection{\label{sec::cco} Compute $C_0$}

$C_0$ is calculated by the formula
\begin{equation}
C_0 = \frac{\left | \rho g \thk \nabla{\usrf^{*}}  \right |}{ \left | u_{\rm obs}^{*}  \right | }
\end{equation}
where $\usrf^{*}$ and $u_{\rm obs}^{*}$ are derived from the observed speed $\left| u_{\rm obs} \right |$
and the surface $\usrf$ computed from the topography and thickness.

As the raw surface is noisy, numerical differentation is ill-posed, so we create a smoothed
surface $\usrf^{*}$ by solving
\begin{equation}
\usrf^{*} - \lambda \nabla^2 \usrf^{*} = \usrf \qquad \lambda = 8 \mbox{~km}
\end{equation}
with natural boundary conditions

The observed speed $\left| u_{\rm obs} \right |$ has a number of missing regions.
We need values everywhere to compute $C_0$. We grow them by repeatedly averaging
the values in cells with $\left| u_{\rm obs} \right | < 1$  from neighbour cells 
that have $\left| u_{\rm obs} \right | > 1$. We do not need these grown regions in the control problem proper. 

\subsection{\label{sec::rri} Remove remote ice}

Ice that is not connected to grounded ice on the main continental mass 
is identified and removed

\section{Control Problem}


\subsection{Test run 001}

Test run 001 was run on May 31 2013, with finest resolutions of 4, 2 , and 1 km.
Preprocessing consisted of \ref{sec::rtt}, \ref{sec::fts}, \ref{sec::cco}, \ref{sec::rri}

\begin{verbatim}
#control problem parameters
control.velMisfitCoefficient = 1.0
control.massImbalanceCoefficient = 0.0e+2
control.gradCsqRegularization = 1.0e+0
control.gradMuCoefsqRegularization = 0.0e+3
control.muCoefLEQOne = FALSE
control.X0Regularization = 1.0e+3
control.X1Regularization = 1.0e+5
control.boundArgX0 = 5.0
control.boundArgX1 = 2.5
control.writeInnerSteps = true
control.outerStepFileNameBase = antarcticabm2-ctrl-4lev-outer.
control.innerStepFileNameBase = antarcticabm2-ctrl-4lev-inner.
control.CGsecantStepMaxGrow = 2.0
control.initialMu = 1.0e+6
control.CGmaxIter = 32
control.CGtol = 1.0e-6
control.CGsecantParameter = 1.0e-8
control.CGsecantMaxIter = 20
control.CGsecantTol = 1.0e-1
amr.max_box_size = 64
amr.max_level = 4
amr.refinement_ratios = 2 2 2 2 2 2 
amr.blockFactor = 8          # block factor used in grid generation
amr.fillRatio = 0.8          # how efficient are the grids
amr.nestingRadius = 1        # proper nesting radius required
amr.tags_grow = 1            # amount o buffer tags
amr.max_vel_dx = 1.0e+6
\end{verbatim}

Blue crystal 2 managed about 1 CG iteration per hour, on 8,16,32 CPUs for the 
4, 2 , and 1 km problems. The 1 km problem is of most interest

\begin{center}
\includegraphics[width=.75\textwidth]{Antarctica2013-all-control.gfx/AMR0000}
\end{center}

After 6 CG iterations, residual L2-norm is reduced from 8e16 to 1e16. 
Velocity field looks smooth, and about right in most of the domain (as it does from the start)

\begin{center}
\includegraphics[width=.75\textwidth]{Antarctica2013-all-control.gfx/AMR0001}
\end{center}

Enhancement factor (muCoef) showing usual pattern in ASE, Filchner-Ronne, with
weak shear margins and stiffer shelves. 

\begin{center}
\includegraphics[width=.75\textwidth]{Antarctica2013-all-control.gfx/AMR0002}
\end{center}

Extreme stiffenning in Amery near its Southern end, epsecially at 'lower' (East) wall (scale
is satuarated). This seems to be an artifact generated to reduce spurious high speed there


\begin{center}
\includegraphics[width=.75\textwidth]{Antarctica2013-all-control.gfx/AMR0004}
\end{center}

Velocity vectors pointing across the stream : shelf thins toward edge - is this an artifact
of preprocessing from section (1), ie we made the ice to thin to avoid grouding it. If
so, better to ground it. Amery also has a noticable discontinuity in the surface elevation at the
GL, so maybe we need a firn correction as Anne did for ALBMAP.

\begin{center}
\includegraphics[width=.75\textwidth]{Antarctica2013-all-control.gfx/AMR0005}
\end{center}

Cross stream velocity also apparent in Rutford ice stream: Here ice is falling into
the stream from a higher surface.  An interpolation artifact of some kind, not sure if it is mine
or bedmap2. Might go away with better resolution, but might also need to modify the topgraphy / thickness 
so that the velocity and geoemtry data agree in the position of the stream. 


\begin{center}
\includegraphics[width=.75\textwidth]{Antarctica2013-all-control.gfx/AMR0006}
\end{center}

1 km resolution regions needs to be extended : reduce amr.max\_vel\_dx 


\bibliographystyle{unsrt}
\bibliography{Antarctica2013}
\end{document}
